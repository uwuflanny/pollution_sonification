%
%%%%%%%%%%%%%%%%%%%%%%%%%%%%%%%%%%%%%%%%
\pagenumbering{roman}                   %serve per mettere i numeri romani
\chapter*{Introduzione}                 %crea l'introduzione (un capitolo
                                        %   non numerato)
%%%%%%%%%%%%%%%%%%%%%%%%%%%%%%%%%%%%%%%%%imposta l'intestazione di pagina
\rhead[\fancyplain{}{\bfseries
INTRODUZIONE}]{\fancyplain{}{\bfseries\thepage}}
\lhead[\fancyplain{}{\bfseries\thepage}]{\fancyplain{}{\bfseries
INTRODUZIONE}}
%%%%%%%%%%%%%%%%%%%%%%%%%%%%%%%%%%%%%%%%%aggiunge la voce Introduzione
                                        %   nell'indice
\addcontentsline{toc}{chapter}{Introduzione}
Questo volume di tesi si pone l'obbiettivo di descrivere la progettazione e lo sviluppo di un'applicazione volta ad aumentare la sensibilità sul discorso dell'inquinamento atmosferico.
\\\\
L'applicativo utilizza la tecnica si Sonificazione, per presentare i dati relativi all'argomento in maniera più intuitiva e comprensibile, stimulando sia l'udito che la percezione visiva.
\\\\
In quanto tesi di progetto, il documento si sofferma sulla creazione dell'audio da un punto di vista tecnico, utilizzando comunque un linguaggio comprensibile da non appassionati di musica digitale.
\\\\
La tesi è divisa in tre capitoli principali, il primo dedicato a presentare il contesto in cui si inserisce il progetto, il secondo elenca e descrive le scelte progettuali e le tecnologie utilizzate, mentre il terzo spiega le decisioni implementative prese.


%%%%%%%%%%%%%%%%%%%%%%%%%%%%%%%%%%%%%%%%%non numera l'ultima pagina sinistra
\clearpage{\pagestyle{empty}\cleardoublepage}