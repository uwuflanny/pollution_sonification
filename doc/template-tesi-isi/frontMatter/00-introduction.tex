%
%%%%%%%%%%%%%%%%%%%%%%%%%%%%%%%%%%%%%%%%
\pagenumbering{roman}                   %serve per mettere i numeri romani
\chapter*{Introduzione}                 %crea l'introduzione (un capitolo
                                        %   non numerato)
%%%%%%%%%%%%%%%%%%%%%%%%%%%%%%%%%%%%%%%%%imposta l'intestazione di pagina
\rhead[\fancyplain{}{\bfseries
INTRODUZIONE}]{\fancyplain{}{\bfseries\thepage}}
\lhead[\fancyplain{}{\bfseries\thepage}]{\fancyplain{}{\bfseries
INTRODUZIONE}}
%%%%%%%%%%%%%%%%%%%%%%%%%%%%%%%%%%%%%%%%%aggiunge la voce Introduzione
                                        %   nell'indice
\addcontentsline{toc}{chapter}{Introduzione}

Le problematiche relative al peggioramento della qualità dell'aria influenzano senza dubbio la qualità della vita delle persone.
Vari processi antropici come l'industrializzazione e l'urbanizzazione hanno reso l'aria in varie zone del pianeta una risorsa sempre più degradata e addirittura pericolosa.
\\\\
Seppure la sensibilizzazione dell'opinione pubblica sia cresciuta ultimamente, l'inquinamento è ancora un problema che non viene affrontato con la dovuta importanza, almeno da una parte troppo grande della popolazione.
Analogamente a problematiche simili, la poca attenzione che viene dedicata al peggioramento della qualità dell'aria è dovuta al fatto che questo fenomeno non è immediatamente percepibile, e l'unico modo per verificarne l'esistenza è quello di effettuare misurazioni.
La barriera che separa la popolazione dalla consapevolezza sul fenomeno consiste quindi nelle difficoltà di accesso e comprensione delle informazioni relative all'argomento.
\\\\
Lo scopo di questo elaborato è quello di rendere più accessibili le informazioni relative all'inquinamento atmosferico, affiancando alla informazioni grafiche una rappresentazione uditiva, ottenuta tramite Sonificazione.
La sonificazione è il processo con il quale un dato viene trasformato in un elemento sonoro che ne rispecchia le caratteristiche, aumentandone così la capacità di comunicazione e il coinvolgimento tramite lo stimolo dell'orecchio.
Tramite una rappresentazione acustica, l'utente può avere un'idea più chiara e immediata della situazione, delegando la comprensione delle informazioni a un processo più naturale, soggettivo e interpretativo, caratteristico dell'udito.
La tesi si pone quindi l'obbiettivo di sviluppare un applicativo che permetta di sonificare i dati relativi alla qualità dell'aria, minimizzando le nozioni necessarie all'utente per poter comprendere il contesto e astrattizzando ogni complessità tecnica.
\\\\
Il volume di tesi è suddiviso in tre capitoli:

\begin{itemize}
    \item \textbf{Capitolo 1 - Background e Contesto}: introduce il lettore al contesto in cui il progetto si inserisce, e alle motivazioni che hanno portato alla sua realizzazione.
    Particolare attenzione è riservata a descrivere il problema dell'inquinamento atmosferico, le sue conseguenze e le tecniche tramite le quali questo viene rappresentato.
    Utilizzando un linguaggio non troppo tecnico sarà presentato il concetto di Sonificazione: l'argomento principale del progetto, per poi spiegarne le caratteristiche, le possibili applicazioni, le strategie di realizzazione e alcuni esempi reperibili in rete.
    Verranno inoltre introdotte alcune nozioni sulla produzione audio digitale, che verranno approfondite nei capitoli successivi;
    \item \textbf{Capitolo 2 - Design e Tecnologie}: descrive le scelte tecniche effettuate per la realizzazione del progetto.
    In seguito all'elenco dei requisiti, verranno mostrate alcune schermate relative all'interfaccia utente realizzate in fase di progettazione grafica, e verranno descritte le tecnologie utilizzate per la realizzazione del sistema, giustificandone le scelte.
    Il capitolo presta particolare attenzione a distinguere le tecnologie inarenti alla parte web rispetto a quelle utilizzate per la generazione dell'audio;
    al termine di questo, sarà possibile avere una panoramica completa dell'architettura del sistema;
    \newpage
    \item \textbf{Capitolo 3 - Implementazione}: descrive in dettaglio le scelete implementative prese durante lo sviluppo del sistema.
    Verranno descritte le funzionalità in maniera approfondita: gli argomenti trattati principalmente in questa sottosezione riguardano la realizzazione e la struttura della parte di front-end, e la generazione dell'audio da un punto di vista sia tecnico che musicale.
    Infine, verrà discussa la fase di test effettuata a seguito del completamento del progetto: verrà introdotta la metrica adottata per misurare la qualità del sistema, verranno mostrati i risultati ottenuti e saranno elencati i problemi riscontrati durante questa fase.
\end{itemize}

%%%%%%%%%%%%%%%%%%%%%%%%%%%%%%%%%%%%%%%%%non numera l'ultima pagina sinistra
\clearpage{\pagestyle{empty}\cleardoublepage}
