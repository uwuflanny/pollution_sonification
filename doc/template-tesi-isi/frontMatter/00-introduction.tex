%
%%%%%%%%%%%%%%%%%%%%%%%%%%%%%%%%%%%%%%%%
\pagenumbering{roman}                   %serve per mettere i numeri romani
\chapter*{Introduzione}                 %crea l'introduzione (un capitolo
                                        %   non numerato)
%%%%%%%%%%%%%%%%%%%%%%%%%%%%%%%%%%%%%%%%%imposta l'intestazione di pagina
\rhead[\fancyplain{}{\bfseries
INTRODUZIONE}]{\fancyplain{}{\bfseries\thepage}}
\lhead[\fancyplain{}{\bfseries\thepage}]{\fancyplain{}{\bfseries
INTRODUZIONE}}
%%%%%%%%%%%%%%%%%%%%%%%%%%%%%%%%%%%%%%%%%aggiunge la voce Introduzione
                                        %   nell'indice
\addcontentsline{toc}{chapter}{Introduzione}
Questo volume di tesi descrive la progettazione e la realizzazione di un sistema volto ad aumentare la consapevolezza sull'inquinamento atmosferico tramite la sonificazione dei dati sulla qualità dell'aria.
\\\\
La sonificazione è il processo con il quale un dato viene trasformato in un elemento sonoro, aumentandone così la capacità di comunicazione e il coinvolgimento tramite lo stimolo dell'udito.
\\\\
Siccome questo sistema dovrà essere utilizzabile da un pubblico non specializzato, verranno forniti alcuni concetti di base riguardanti l'inquinamento e le metriche utilizzate per la misurazione di questo.
Saranno inoltre elencati i rischi per la salute che il peggioramento della qualità dell'aria comporta, e le principali tipologie di rappresentazione grafica dei dati ambienetali.
\\\\
All'interno di questo testo saranno integrati concetti di produzione sonora e audio digitale, al fine di fornire una panoramica completa sul funzionamento del sistema sonificativo.
Per comprendere il collegamento tra un dato e la sua rappresentazione sonora, saranno discusse le metodogie sonificative solitamente utilizzate, comprese di esempi pratici.

\newpage

\subsubsection{Suddivisione dei contenuti}
Il volume di tesi è composto da tre capitoli principali, ognuno dei quali diviso in più sezioni.
La spartizione dei contenuti è stata fatta in modo da raggruppare gli argomenti seguendo la logica delle fasi di sviluppo del progetto.

\subsubsection{Primo Capitolo}
Il primo capitolo è dedicato ad introdurre il lettore al contesto in cui il progetto si inserisce, e alle motivazioni che hanno portato alla sua realizzazione.
Particolare attenzione è riservata a descrivere il problema dell'inquinamento atmosferico, le sue conseguenze e le tecniche tramite le quali questo viene rappresentato.
Utilizzando un linguaggio non troppo tecnico sarà presentato il concetto di Sonificazione: l'argomento principale del progetto, per poi spiegarne le caratteristiche, le possibili applicazioni, le strategie di realizzazione e alcuni esempi reperibili in rete.
Verranno inoltre introdotte alcune nozioni sulla produzione audio digitale, che verranno approfondite nei capitoli successivi.

\subsubsection{Secondo Capitolo}
Il secondo capitolo è dedicato alla descrizione delle scelte tecniche effettuate per la realizzazione del progetto.
In seguito all'elenco dei requisiti, verranno mostrate alcune schermate relative all'interfaccia utente realizzate in fase di progettazione grafica, e verranno descritte le tecnologie utilizzate per la realizzazione del sistema, giustificandone le scelte.
Il capitolo presta particolare attenzione a distinguere le tecnologie inarenti alla parte web rispetto a quelle utilizzate per la generazione dell'audio;
al termine di questo, sarà possibile avere una panoramica completa dell'architettura del sistema.

\subsubsection{Terzo Capitolo}
Il terzo capitolo descrive in dettaglio le scelete implementative prese durante lo sviluppo del sistema.
Verranno descritte le funzionalità in maniera approfondita: gli argomenti trattati principalmente in questa sottosezione riguardano la realizzazione e la struttura della parte di front-end, e la generazione dell'audio da un punto di vista sia tecnico che musicale.
Infine, verrà discussa la fase di test effettuata a seguito del completamento del progetto: verrà introdotta la metrica adottata per misurare la qualità del sistema, verranno mostrati i risultati ottenuti saranno elencati i problemi riscontrati durante questa fase.



%%%%%%%%%%%%%%%%%%%%%%%%%%%%%%%%%%%%%%%%%non numera l'ultima pagina sinistra
\clearpage{\pagestyle{empty}\cleardoublepage}
