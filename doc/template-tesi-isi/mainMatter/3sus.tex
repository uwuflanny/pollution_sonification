\subsection{Strategia di sonificazione}
Per sonificare i dati, ho utilizzato un approccio misto, sfruttando sia la sintesi audio che il design sonoro, approfonditi nel primo capitolo.
Prima di progettare il sistema sonificativo, ho individuato le caratteristiche dei dati che volevo rappresentare:
le componenti di maggiore importanza sono l’andamento temporale dell'inquinamento e l'inquinamento residuo.
\subsubsection{L'andamento dell'inquinamento}
L'andamento temporale è rappresentato dalle varie rilevazioni nel tempo, siccome è un valore che tende a cambire rapidamente nel tempo, ha richiesto un approccio di sonificazione dinamico.
La scelta delle note utilizzate per rappresentare questo dato è molto semplice: più il valore dell'inquinamento è alto, più la nota è grave.
\subsubsection{L'inquinamento residuo}
L'inquinamento residuo consiste nel valore di rischio per la salute che si protrae nel tempo in seguito ad una giornata particolarmente inquinata.
A differenza della rilevazione AQI vera e propria, i cui valori possono variare drasticamente da un momento all'altro, l'accumulo di residuo si distingue in fasi di crescita e di diminuzione, in funzione a come si è comportato l'inquinamento nel corso della giornata \cite{residue}.
Per le fasi di crescita ho usato delle note veloci che si alzano di tonalità, mentre per la fase di diminuzione le stesse note sono ripetute con meno frequenza, e tendono ad abbassarsi nel tempo.
\subsubsection{Gestire il volume}
Un aspetto molto rilevante che ho attribuito a queste due tracce sta nella miscela dei loro volumi: più il valore dell'inquinamento tende ad alzarsi, più il suo volume si abbassa per lasciare spazio alla traccia dell'inquinamento residuo, che invece tende a crescere.
\subsubsection{L'applicazione della teoria musicale}
Per accentuare l'effetto della crescita e della diminuzione dell'inquinamento residuo, le note della traccia seguono una scala: più l'inquinamento è alto più le note sono acute.
Ho deciso di utilizzare una scala maggiore. Siccome le tonalità maggiori sono note per la loro positività, ho introdotto delle dissonanze in base al peggioramento della qualità dell'aria, volte a spezzare questa armonia.
Una dissonanza è una nota “fuori contesto” rispetto alla scala o all'accordo che si sta suonando; se inserita all'interno di una scala maggiore, sempre positiva e armoniosa, la dissonanza crea un effetto improvviso di tensione \cite{dissonance}.
Ho posto le dissonanze ad intervalli regolari nella scala in base al livello di inquinamento: più questo è alto, più le dissonanze sono frequenti e di maggior intensità.